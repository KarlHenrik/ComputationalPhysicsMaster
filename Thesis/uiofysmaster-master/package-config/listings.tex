\definecolor{listingsstringcolor}{rgb}{0,0.5,0}
\definecolor{listingskeywordcolor}{rgb}{0.5,0.5,0.0}
\definecolor{listingsbasiccolor}{rgb}{0.0,0.0,0.0}
% \definecolor{listingskeywordcolor}{rgb}{0.0,0.0,0.7}
\definecolor{listingsnumbercolor}{rgb}{0.0,0.0,1.0}
\definecolor{listingscommentcolor}{rgb}{0.4,0.4,0.4}
\definecolor{listingsbackgroundcolor}{rgb}{0.975,0.975,0.975}
\definecolor{listingsrulecolor}{rgb}{0.86,0.86,0.86}
\definecolor{listingsidentifiercolor}{rgb}{0.0,0.0,0.0}
\definecolor{listingsclasscolor}{rgb}{0.5,0.0,0.5}
\definecolor{listingsmembercolor}{rgb}{0.5,0.0,0.0}
\definecolor{listingsdirectivecolor}{rgb}{0.0,0.0,0.5}
% \definecolor{listingsvariablecolor}{rgb}{0.5,0.0,0.5}

\RequirePackage{iftex}

\newcommand{\listingsfont}{\sffamily}
\lstset {
    language=c++,
    numbers=none,
    breaklines=true,
    tabsize=2,
    backgroundcolor=\color{listingsbackgroundcolor},
    breakatwhitespace=true,         % sets if automatic breaks should only happen at whitespace
    breaklines=true,                 % sets automatic line breaking
  %   numbers=left,                    % where to put the line-numbers; possible values are (none, left, right)
    numbersep=5pt,                   % how far the line-numbers are from the code
    frame=lrtb,                    % adds a frame around the code
    framexleftmargin=7pt,
    framexrightmargin=7pt,
    framextopmargin=7pt,
    framexbottommargin=7pt,
    xleftmargin=18pt,
    xrightmargin=18pt,
    rulecolor=\color{listingsrulecolor},
    tabsize=2,                       % sets default tabsize to 2 spaces
    literate={-}{{\textendash}}1 {å}{{\aa}}1 {æ}{{\ae}}1 {ø}{{\oslash}}1,
    showstringspaces=false,
    captionpos=b,
    basicstyle=\color{listingsbasiccolor}\footnotesize\listingsfont,
    keywordstyle=\color{listingskeywordcolor}\footnotesize\listingsfont,
    directivestyle=\color{listingsdirectivecolor}\footnotesize\listingsfont,
    stringstyle=\color{listingsstringcolor}\footnotesize\listingsfont,
    commentstyle=\color{listingscommentcolor}\footnotesize\listingsfont,
    numberstyle=\color{listingsnumbercolor}\footnotesize\listingsfont,
    identifierstyle=\color{listingsidentifiercolor}\footnotesize\listingsfont,
    keywordstyle=[2]{\color{listingsclasscolor}\footnotesize\listingsfont},
    keywordstyle=[3]{\color{listingsmembercolor}\footnotesize\listingsfont},
    keywordstyle=[4]{\color{listingsdirectivecolor}\footnotesize\listingsfont},
}
\lstdefinelanguage{qmake}
{
    sensitive=true,
    morekeywords=[1]{
        absolute_path,
        basename,
        cat,
        clean_path,
        dirname,
        enumerate_vars,
        escape_expand,
        find,
        first,
        format_number,
        fromfile,
        getenv,
        join,
        last,
        list,
        lower,
        member,
        prompt,
        quote,
        re_escape,
        relative_path,
        replace,
        sprintf,
        resolve_depends,
        reverse,
        section,
        shadowed,
        shell_path,
        shell_quote,
        size,
        sort_depends,
        split,
        system,
        system_path,
        system_quote,
        unique,
        upper,
        val_escape,cache,
        contains,
        count,
        debug,
        defined,
        equals,
        error,
        eval,
        exists,
        export,
        files,
        for,
        greaterThan,
        if,
        include,
        infile,
        isActiveConfig,
        isEmpty,
        isEqual,
        lessThan,
        load,
        log,
        message,
        mkpath,
        requires,
        system,
        touch,
        unset,
        warning,
        write_file,
        packagesExist,
        prepareRecursiveTarget,
        qtCompileTest,
        qtHaveModule
    },
    morekeywords=[2]{
            CONFIG,
            DEFINES,
            DEF_FILE,
            DEPENDPATH,
            DEPLOYMENT,
            DEPLOYMENT_PLUGIN,
            DESTDIR,
            DISTFILES,
            DLLDESTDIR,
            FORMS,
            GUID,
            HEADERS,
            ICON,
            INCLUDEPATH,
            INSTALLS,
            LEXIMPLS,
            LEXOBJECTS,
            LEXSOURCES,
            LIBS,
            LITERAL_HASH,
            MAKEFILE,
            MAKEFILE_GENERATOR,
            MOC_DIR,
            OBJECTS,
            OBJECTS_DIR,
            POST_TARGETDEPS,
            PRE_TARGETDEPS,
            PRECOMPILED_HEADER,
            PWD,
            OUT_PWD,
            QMAKE,
            QMAKESPEC,
            QMAKE_AR_CMD,
            QMAKE_BUNDLE_DATA,
            QMAKE_BUNDLE_EXTENSION,
            QMAKE_CC,
            QMAKE_CFLAGS_DEBUG,
            QMAKE_CFLAGS_RELEASE,
            QMAKE_CFLAGS_SHLIB,
            QMAKE_CFLAGS_THREAD,
            QMAKE_CFLAGS_WARN_OFF,
            QMAKE_CFLAGS_WARN_ON,
            QMAKE_CLEAN,
            QMAKE_CXX,
            QMAKE_CXXFLAGS,
            QMAKE_CXXFLAGS_DEBUG,
            QMAKE_CXXFLAGS_RELEASE,
            QMAKE_CXXFLAGS_SHLIB,
            QMAKE_CXXFLAGS_THREAD,
            QMAKE_CXXFLAGS_WARN_OFF,
            QMAKE_CXXFLAGS_WARN_ON,
            QMAKE_DISTCLEAN,
            QMAKE_EXTENSION_SHLIB,
            QMAKE_EXT_MOC,
            QMAKE_EXT_UI,
            QMAKE_EXT_PRL,
            QMAKE_EXT_LEX,
            QMAKE_EXT_YACC,
            QMAKE_EXT_OBJ,
            QMAKE_EXT_CPP,
            QMAKE_EXT_H,
            QMAKE_EXTRA_COMPILERS,
            QMAKE_EXTRA_TARGETS,
            QMAKE_FAILED_REQUIREMENTS,
            QMAKE_FRAMEWORK_BUNDLE_NAME,
            QMAKE_FRAMEWORK_VERSION,
            QMAKE_INCDIR,
            QMAKE_INCDIR_EGL,
            QMAKE_INCDIR_OPENGL,
            QMAKE_INCDIR_OPENGL_ES2,
            QMAKE_INCDIR_OPENVG,
            QMAKE_INCDIR_X11,
            QMAKE_INFO_PLIST,
            QMAKE_LFLAGS,
            QMAKE_LFLAGS_CONSOLE,
            QMAKE_LFLAGS_DEBUG,
            QMAKE_LFLAGS_PLUGIN,
            QMAKE_LFLAGS_RPATH,
            QMAKE_LFLAGS_RPATHLINK,
            QMAKE_LFLAGS_RELEASE,
            QMAKE_LFLAGS_APP,
            QMAKE_LFLAGS_SHLIB,
            QMAKE_LFLAGS_SONAME,
            QMAKE_LFLAGS_THREAD,
            QMAKE_LFLAGS_WINDOWS,
            QMAKE_LIBDIR,
            QMAKE_LIBDIR_FLAGS,
            QMAKE_LIBDIR_EGL,
            QMAKE_LIBDIR_OPENGL,
            QMAKE_LIBDIR_OPENVG,
            QMAKE_LIBDIR_X11,
            QMAKE_LIBS,
            QMAKE_LIBS_EGL,
            QMAKE_LIBS_OPENGL,
            QMAKE_LIBS_OPENGL_ES1, QMAKE_LIBS_OPENGL_ES2,
            QMAKE_LIBS_OPENVG,
            QMAKE_LIBS_THREAD,
            QMAKE_LIBS_X11,
            QMAKE_LIB_FLAG,
            QMAKE_LINK_SHLIB_CMD,
            QMAKE_LN_SHLIB,
            QMAKE_POST_LINK,
            QMAKE_PRE_LINK,
            QMAKE_PROJECT_NAME,
            QMAKE_MAC_SDK,
            QMAKE_MACOSX_DEPLOYMENT_TARGET,
            QMAKE_MAKEFILE,
            QMAKE_QMAKE,
            QMAKE_RESOURCE_FLAGS,
            QMAKE_RPATHDIR,
            QMAKE_RPATHLINKDIR,
            QMAKE_RUN_CC,
            QMAKE_RUN_CC_IMP,
            QMAKE_RUN_CXX,
            QMAKE_RUN_CXX_IMP,
            QMAKE_TARGET,
            QT,
            QTPLUGIN,
            QT_VERSION,
            QT_MAJOR_VERSION,
            QT_MINOR_VERSION,
            QT_PATCH_VERSION,
            RC_FILE,
            RC_INCLUDEPATH,
            RCC_DIR,
            REQUIRES,
            RESOURCES,
            RES_FILE,
            SIGNATURE_FILE,
            SOURCES,
            SUBDIRS,
            TARGET,
            TARGET_EXT,
            TARGET_x,
            TARGET_x.y.z,
            TEMPLATE,
            TRANSLATIONS,
            UI_DIR,
            VERSION,
            VER_MAJ,
            VER_MIN,
            VER_PAT,
            VPATH,
            WINRT_MANIFEST,
            YACCSOURCES,
            _PRO_FILE_,
            _PRO_FILE_PWD_
    },
}
\lstdefinelanguage{GLSL}
{
sensitive=true,
morekeywords=[1]{
attribute, const, uniform, varying,
layout, centroid, flat, smooth,
noperspective, break, continue, do,
for, while, switch, case, default, if,
else, in, out, inout, float, int, void,
bool, true, false, invariant, discard,
return, mat2, mat3, mat4, mat2x2, mat2x3,
mat2x4, mat3x2, mat3x3, mat3x4, mat4x2,
mat4x3, mat4x4, vec2, vec3, vec4, ivec2,
ivec3, ivec4, bvec2, bvec3, bvec4, uint,
uvec2, uvec3, uvec4, lowp, mediump, highp,
precision, sampler1D, sampler2D, sampler3D,
samplerCube, sampler1DShadow,
sampler2DShadow, samplerCubeShadow,
sampler1DArray, sampler2DArray,
sampler1DArrayShadow, sampler2DArrayShadow,
isampler1D, isampler2D, isampler3D,
isamplerCube, isampler1DArray,
isampler2DArray, usampler1D, usampler2D,
usampler3D, usamplerCube, usampler1DArray,
usampler2DArray, sampler2DRect,
sampler2DRectShadow, isampler2DRect,
usampler2DRect, samplerBuffer,
isamplerBuffer, usamplerBuffer, sampler2DMS,
isampler2DMS, usampler2DMS,
sampler2DMSArray, isampler2DMSArray,
usampler2DMSArray, struct},
morekeywords=[2]{
radians,degrees,sin,cos,tan,asin,acos,atan,
atan,sinh,cosh,tanh,asinh,acosh,atanh,pow,
exp,log,exp2,log2,sqrt,inversesqrt,abs,sign,
floor,trunc,round,roundEven,ceil,fract,mod,modf,
min,max,clamp,mix,step,smoothstep,isnan,isinf,
floatBitsToInt,floatBitsToUint,intBitsToFloat,
uintBitsToFloat,length,distance,dot,cross,
normalize,faceforward,reflect,refract,
matrixCompMult,outerProduct,transpose,
determinant,inverse,lessThan,lessThanEqual,
greaterThan,greaterThanEqual,equal,notEqual,
any,all,not,textureSize,texture,textureProj,
textureLod,textureOffset,texelFetch,
texelFetchOffset,textureProjOffset,
textureLodOffset,textureProjLod,
textureProjLodOffset,textureGrad,
textureGradOffset,textureProjGrad,
textureProjGradOffset,texture1D,texture1DProj,
texture1DProjLod,texture2D,texture2DProj,
texture2DLod,texture2DProjLod,texture3D,
texture3DProj,texture3DLod,texture3DProjLod,
textureCube,textureCubeLod,shadow1D,shadow2D,
shadow1DProj,shadow2DProj,shadow1DLod,
shadow2DLod,shadow1DProjLod,shadow2DProjLod,
dFdx,dFdy,fwidth,noise1,noise2,noise3,noise4,
EmitVertex,EndPrimitive
},
morekeywords=[3]{
gl_In, gl_InvocationID,
gl_VertexID,gl_InstanceID,gl_Position,
gl_PointSize,gl_ClipDistance,gl_PerVertex,
gl_Layer,gl_ClipVertex,gl_FragCoord,
gl_FrontFacing,gl_ClipDistance,gl_FragColor,
gl_FragData,gl_MaxDrawBuffers,gl_FragDepth,
gl_PointCoord,gl_PrimitiveID,
gl_MaxVertexAttribs,gl_MaxVertexUniformComponents,
gl_MaxVaryingFloats,gl_MaxVaryingComponents,
gl_MaxVertexOutputComponents,
gl_MaxGeometryInputComponents,
gl_MaxGeometryOutputComponents,
gl_MaxFragmentInputComponents,
gl_MaxVertexTextureImageUnits,
gl_MaxCombinedTextureImageUnits,
gl_MaxTextureImageUnits,
gl_MaxFragmentUniformComponents,
gl_MaxDrawBuffers,gl_MaxClipDistances,
gl_MaxGeometryTextureImageUnits,
gl_MaxGeometryOutputVertices,
gl_MaxGeometryOutputVertices,
gl_MaxGeometryTotalOutputComponents,
gl_MaxGeometryUniformComponents,
gl_MaxGeometryVaryingComponents,gl_DepthRange},
morecomment=[l]{//},
morecomment=[s]{/*}{*/},
morecomment=[l]{\#},
}
\lstdefinelanguage[std]{c++}[ISO]{c++}
{
    morekeywords=[2]{
        array,deque,forward_list,list,map,queue,set,stack,unordered_map,unordered_set,vector,
        unordered_multimap,multimap,multiset,
        basic_istringstream,basic_ostringstream,basic_stringstream,basic_stringbuf,
        istringstream,ostringstream,stringstream,stringbuf,wstringstream,wostringstream,wstringstream,wstringbuf,
        fstream,iomanip,ios,iosfwd,istream,ostream,sstream,streambuf,
        basic_ios,fpos,ios_base,io_errc,streamoff,streampos,streamsize,wstreampos,
        promise,packaged_task,future,shared_future,future_error,future_errc,future_status,launch,
        basic_ifstream,basic_ofstream,basic_fstream,basic_filebuf,
        ifstream,ofstream,fstream,filebuf,wifstream,wofstream,wfstream,wfilebuf,
        atomic,condition_variable,future,mutex,thread,
        algorithm,
        bitset,
        chrono,
        codecvt,
        complex,
        exception,
        functional,
        initializer_list,
        iterator,
        limits,
        locale,
        memory,
        new,
        numeric,
        random,
        ratio,
        regex,smatch,
        stdexcept,
        string,
        system_error,
        tuple,
        typeindex,
        typeinfo,
        type_traits,
        utility,
        valarray,
    },
    morecomment=[l][keywordstyle4]{\#include},
}


\lstdefinelanguage[armadillo]{c++}[std]{c++}
{
    morekeywords=[2]{
        Mat,mat,cx_mat,Col,colvec,vec,Row,rowvec,Cube,cube,field,SpMat,sp_mat,sp_cx_mat
    },
}


\lstdefinelanguage[qt]{c++}[std]{c++}
{
    morekeywords=[2]{
        QGL,
        QGLPainter,
        DrawingMode,
        Points,
        Position,
        QAbstractDownloadManager,
        QArray,
        QBox3D,
        QColor4ub,
        QCustomDataArray,
        QDownloadManager,
        QGeometryData,
        QGLAbstractEffect,
        QGLAbstractMaterial,
        QGLAbstractScene,
        QGLAbstractSurface,
        QGLAttributeDescription,
        QGLAttributeSet,
        QGLAttributeValue,
        QGLBezierPatches,
        QGLBuilder,
        QGLCamera,
        QGLCameraAnimation,
        QGLColladaFxEffectLoader,
        QGLColorMaterial,
        QGLCube,
        QGLCylinder,
        QGLDome,
        QGLFramebufferObjectSurface,
        QGLGraphicsViewportItem,
        QGLIndexBuffer,
        QGLLightModel,
        QGLLightParameters,
        QGLMaterial,
        QGLMaterialCollection,
        QGLPainter,
        QGLPickNode,
        QGLPixelBufferSurface,
        QGLRenderOrder,
        QGLRenderOrderComparator,
        QGLRenderSequencer,
        QGLRenderState,
        QGLSceneFormatHandler,
        QGLSceneFormatPlugin,
        QGLSceneNode,
        QGLShaderProgramEffect,
        QGLSphere,
        QGLSubsurface,
        QGLTeapot,
        QGLTexture2D,
        QGLTextureCube,
        QGLTwoSidedMaterial,
        QGLVertexBundle,
        QGLView,
        QGLWidgetSurface,
        QGraphicsBillboardTransform,
        QGraphicsEmbedScene,
        QGraphicsLookAtTransform,
        QGraphicsRotation3D,
        QGraphicsScale3D,
        QGraphicsTransform3D,
        QGraphicsTranslation3D,
        QLogicalVertex,
        QMatrix4x4Stack,
        QOpenGLFunctions,
        QPlane3D,
        QRay3D,
        QSphere3D,
        QThreadedDownloadManager,
        QTriangle3D,
        QVector2DArray,
        QVector3DArray,
        QVector4DArray ,
        QAbstractAnimation,
        QAbstractButton,
        QAbstractEventDispatcher,
        QAbstractExtensionFactory,
        QAbstractExtensionManager,
        QAbstractFileEngine,
        QAbstractFileEngineHandler,
        QAbstractFileEngineIterator,
        QAbstractFontEngine,
        QAbstractFormBuilder,
        QAbstractGraphicsShapeItem,
        QAbstractItemDelegate,
        QAbstractItemModel,
        QAbstractItemView,
        QAbstractListModel,
        QAbstractMessageHandler,
        QAbstractNetworkCache,
        QAbstractPrintDialog,
        QAbstractProxyModel,
        QAbstractScrollArea,
        QAbstractSlider,
        QAbstractSocket,
        QAbstractSpinBox,
        QAbstractState,
        QAbstractTableModel,
        QAbstractTextDocumentLayout,
        QAbstractTransition,
        QAbstractUriResolver,
        QAbstractVideoBuffer,
        QAbstractVideoSurface,
        QAbstractXmlNodeModel,
        QAbstractXmlReceiver,
        QAccessible,
        QAccessibleBridge,
        QAccessibleBridgePlugin,
        QAccessibleEvent,
        QAccessibleInterface,
        QAccessibleObject,
        QAccessiblePlugin,
        QAccessibleWidget,
        QAction,
        QActionEvent,
        QActionGroup,
        QAnimationGroup,
        QApplication,
        QAtomicInt,
        QAtomicPointer,
        AudioDataOutput (Phonon),
        QAudioDeviceInfo,
        QAudioFormat,
        QAudioInput,
        QAudioOutput,
        QAuthenticator,
        QAxAggregated,
        QAxBase,
        QAxBindable,
        QAxFactory,
        QAxObject,
        QAxScript,
        QAxScriptEngine,
        QAxScriptManager,
        QAxWidget,
        QBasicTimer,
        QBitArray,
        QBitmap,
        QBoxLayout,
        QBrush,
        QBuffer,
        QButtonGroup,
        QByteArray,
        QByteArrayMatcher,
        QCache,
        QCalendarWidget,
        QCDEStyle,
        QChar,
        QCheckBox,
        QChildEvent,
        QCleanlooksStyle,
        QClipboard,
        QCloseEvent,
        QColor,
        QColorDialog,
        QColormap,
        QColumnView,
        QComboBox,
        QCommandLinkButton,
        QCommonStyle,
        QCompleter,
        QConicalGradient,
        QContextMenuEvent,
        QContiguousCache,
        QCopChannel,
        QCoreApplication,
        QCryptographicHash,
        QCursor,
        QCustomRasterPaintDevice,
        QDataStream,
        QDataWidgetMapper,
        QDate,
        QDateEdit,
        QDateTime,
        QDateTimeEdit,
        QDBusAbstractAdaptor,
        QDBusAbstractInterface,
        QDBusArgument,
        QDBusConnection,
        QDBusConnectionInterface,
        QDBusContext,
        QDBusError,
        QDBusInterface,
        QDBusMessage,
        QDBusObjectPath,
        QDBusPendingCall,
        QDBusPendingCallWatcher,
        QDBusPendingReply,
        QDBusReply,
        QDBusServiceWatcher,
        QDBusSignature,
        QDBusUnixFileDescriptor,
        QDBusVariant,
        QDebug,
        QDeclarativeComponent,
        QDeclarativeContext,
        QDeclarativeEngine,
        QDeclarativeError,
        QDeclarativeExpression,
        QDeclarativeExtensionPlugin,
        QDeclarativeImageProvider,
        QDeclarativeItem,
        QDeclarativeListProperty,
        QDeclarativeListReference,
        QDeclarativeNetworkAccessManagerFactory,
        QDeclarativeParserStatus,
        QDeclarativeProperty,
        QDeclarativePropertyMap,
        QDeclarativePropertyValueSource,
        QDeclarativeScriptString,
        QDeclarativeView,
        QDecoration,
        QDecorationDefault,
        QDecorationFactory,
        QDecorationPlugin,
        QDesignerActionEditorInterface,
        QDesignerContainerExtension,
        QDesignerCustomWidgetCollectionInterface,
        QDesignerCustomWidgetInterface,
        QDesignerDynamicPropertySheetExtension,
        QDesignerFormEditorInterface,
        QDesignerFormWindowCursorInterface,
        QDesignerFormWindowInterface,
        QDesignerFormWindowManagerInterface,
        QDesignerMemberSheetExtension,
        QDesignerObjectInspectorInterface,
        QDesignerPropertyEditorInterface,
        QDesignerPropertySheetExtension,
        QDesignerTaskMenuExtension,
        QDesignerWidgetBoxInterface,
        QDesktopServices,
        QDesktopWidget,
        QDial,
        QDialog,
        QDialogButtonBox,
        QDir,
        QDirectPainter,
        QDirIterator,
        QDockWidget,
        QDomAttr,
        QDomCDATASection,
        QDomCharacterData,
        QDomComment,
        QDomDocument,
        QDomDocumentFragment,
        QDomDocumentType,
        QDomElement,
        QDomEntity,
        QDomEntityReference,
        QDomImplementation,
        QDomNamedNodeMap,
        QDomNode,
        QDomNodeList,
        QDomNotation,
        QDomProcessingInstruction,
        QDomText,
        QDoubleSpinBox,
        QDoubleValidator,
        QDrag,
        QDragEnterEvent,
        QDragLeaveEvent,
        QDragMoveEvent,
        QDropEvent,
        QDynamicPropertyChangeEvent,
        QEasingCurve,
        Effect (Phonon),
        EffectParameter (Phonon),
        EffectWidget (Phonon),
        QElapsedTimer,
        QErrorMessage,
        QEvent,
        QEventLoop,
        QEventTransition,
        Exception (QtConcurrent),
        QExplicitlySharedDataPointer,
        QExtensionFactory,
        QExtensionManager,
        QFile,
        QFileDialog,
        QFileIconProvider,
        QFileInfo,
        QFileOpenEvent,
        QFileSystemModel,
        QFileSystemWatcher,
        QFinalState,
        QFlag,
        QFlags,
        QFocusEvent,
        QFocusFrame,
        QFont,
        QFontComboBox,
        QFontDatabase,
        QFontDialog,
        QFontEngineInfo,
        QFontEnginePlugin,
        QFontInfo,
        QFontMetrics,
        QFontMetricsF,
        QFormBuilder,
        QFormLayout,
        QFrame,
        QFSFileEngine,
        QFtp,
        QFuture,
        QFutureIterator,
        QFutureSynchronizer,
        QFutureWatcher,
        QGenericArgument,
        QGenericMatrix,
        QGenericPlugin,
        QGenericPluginFactory,
        QGenericReturnArgument,
        QGesture,
        QGestureEvent,
        QGestureRecognizer,
        QGLBuffer,
        QGLColormap,
        QGLContext,
        QGLFormat,
        QGLFramebufferObject,
        QGLFramebufferObjectFormat,
        QGLFunctions,
        QGLPixelBuffer,
        QGLShader,
        QGLShaderProgram,
        QGLWidget,
        QGlyphRun,
        QGradient,
        QGraphicsAnchor,
        QGraphicsAnchorLayout,
        QGraphicsBlurEffect,
        QGraphicsColorizeEffect,
        QGraphicsDropShadowEffect,
        QGraphicsEffect,
        QGraphicsEllipseItem,
        QGraphicsGridLayout,
        QGraphicsItem,
        QGraphicsItemAnimation,
        QGraphicsItemGroup,
        QGraphicsLayout,
        QGraphicsLayoutItem,
        QGraphicsLinearLayout,
        QGraphicsLineItem,
        QGraphicsObject,
        QGraphicsOpacityEffect,
        QGraphicsPathItem,
        QGraphicsPixmapItem,
        QGraphicsPolygonItem,
        QGraphicsProxyWidget,
        QGraphicsRectItem,
        QGraphicsRotation,
        QGraphicsScale,
        QGraphicsScene,
        QGraphicsSceneContextMenuEvent,
        QGraphicsSceneDragDropEvent,
        QGraphicsSceneEvent,
        QGraphicsSceneHelpEvent,
        QGraphicsSceneHoverEvent,
        QGraphicsSceneMouseEvent,
        QGraphicsSceneMoveEvent,
        QGraphicsSceneResizeEvent,
        QGraphicsSceneWheelEvent,
        QGraphicsSimpleTextItem,
        QGraphicsSvgItem,
        QGraphicsTextItem,
        QGraphicsTransform,
        QGraphicsView,
        QGraphicsWebView,
        QGraphicsWidget,
        QGridLayout,
        QGroupBox,
        QGtkStyle,
        QHash,
        QHashIterator,
        QHBoxLayout,
        QHeaderView,
        QHelpContentItem,
        QHelpContentModel,
        QHelpContentWidget,
        QHelpEngine,
        QHelpEngineCore,
        QHelpEvent,
        QHelpIndexModel,
        QHelpIndexWidget,
        QHelpSearchEngine,
        QHelpSearchQuery,
        QHelpSearchQueryWidget,
        QHelpSearchResultWidget,
        QHideEvent,
        QHistoryState,
        QHostAddress,
        QHostInfo,
        QHoverEvent,
        QHttpMultiPart,
        QHttpPart,
        QIcon,
        QIconDragEvent,
        QIconEngine,
        QIconEnginePlugin,
        QIconEnginePluginV2,
        QIconEngineV2,
        QIdentityProxyModel,
        QImage,
        QImageIOHandler,
        QImageIOPlugin,
        QImageReader,
        QImageWriter,
        QInputContext,
        QInputContextFactory,
        QInputContextPlugin,
        QInputDialog,
        QInputEvent,
        QInputMethodEvent,
        QIntValidator,
        QIODevice,
        QItemDelegate,
        QItemEditorCreator,
        QItemEditorCreatorBase,
        QItemEditorFactory,
        QItemSelection,
        QItemSelectionModel,
        QItemSelectionRange,
        QKbdDriverFactory,
        QKbdDriverPlugin,
        QKeyEvent,
        QKeyEventTransition,
        QKeySequence,
        QLabel,
        QLatin1Char,
        QLatin1String,
        QLayout,
        QLayoutItem,
        QLCDNumber,
        QLibrary,
        QLibraryInfo,
        QLine,
        QLinearGradient,
        QLineEdit,
        QLineF,
        QLinkedList,
        QLinkedListIterator,
        QList,
        QListIterator,
        QListView,
        QListWidget,
        QListWidgetItem,
        QLocale,
        QLocalServer,
        QLocalSocket,
        QMacCocoaViewContainer,
        QMacNativeWidget,
        QMacPasteboardMime,
        QMacStyle,
        QMainWindow,
        QMap,
        QMapIterator,
        QMargins,
        QMatrix4x4,
        QMdiArea,
        QMdiSubWindow,
        MediaController (Phonon),
        MediaNode (Phonon),
        MediaObject (Phonon),
        MediaSource (Phonon),
        QMenu,
        QMenuBar,
        QMessageBox,
        QMetaClassInfo,
        QMetaEnum,
        QMetaMethod,
        QMetaObject,
        QMetaProperty,
        QMetaType,
        QMimeData,
        QModelIndex,
        QMotifStyle,
        QMouseDriverFactory,
        QMouseDriverPlugin,
        QMouseEvent,
        QMouseEventTransition,
        QMoveEvent,
        QMovie,
        QMultiHash,
        QMultiMap,
        QMutableHashIterator,
        QMutableLinkedListIterator,
        QMutableListIterator,
        QMutableMapIterator,
        QMutableSetIterator,
        QMutableVectorIterator,
        QMutex,
        QMutexLocker,
        QNetworkAccessManager,
        QNetworkAddressEntry,
        QNetworkCacheMetaData,
        QNetworkConfiguration,
        QNetworkConfigurationManager,
        QNetworkCookie,
        QNetworkCookieJar,
        QNetworkDiskCache,
        QNetworkInterface,
        QNetworkProxy,
        QNetworkProxyFactory,
        QNetworkProxyQuery,
        QNetworkReply,
        QNetworkRequest,
        QNetworkSession,
        Notifier (Phonon::BackendCapabilities),
        QObject,
        QObjectCleanupHandler,
        ObjectDescription (Phonon),
        QPageSetupDialog,
        QPaintDevice,
        QPaintEngine,
        QPaintEngineState,
        QPainter,
        QPainterPath,
        QPainterPathStroker,
        QPaintEvent,
        QPair,
        QPalette,
        QPanGesture,
        QParallelAnimationGroup,
        Path (Phonon),
        QPauseAnimation,
        QPen,
        QPersistentModelIndex,
        QPicture,
        QPinchGesture,
        QPixmap,
        QPixmapCache,
        QPlainTextDocumentLayout,
        QPlainTextEdit,
        QPlastiqueStyle,
        QPlatformCursor,
        QPlatformCursorImage,
        QPlatformFontDatabase,
        QPlatformWindowFormat,
        QPluginLoader,
        QPoint,
        QPointer,
        QPointF,
        QPolygon,
        QPolygonF,
        QPrintDialog,
        QPrintEngine,
        QPrinter,
        QPrinterInfo,
        QPrintPreviewDialog,
        QPrintPreviewWidget,
        QProcess,
        QProcessEnvironment,
        QProgressBar,
        QProgressDialog,
        QPropertyAnimation,
        QProxyScreen,
        QProxyScreenCursor,
        QProxyStyle,
        QPushButton,
        QTouchEventSequence (QTest),
        QQuaternion,
        QQueue,
        QRadialGradient,
        QRadioButton,
        QRasterPaintEngine,
        QRawFont,
        QReadLocker,
        QReadWriteLock,
        QRect,
        QRectF,
        QRegExp,
        QRegExpValidator,
        QRegion,
        QResizeEvent,
        QResource,
        QRubberBand,
        QRunnable,
        QS60MainApplication,
        QS60MainAppUi,
        QS60MainDocument,
        QS60Style,
        QScopedArrayPointer,
        QScopedPointer,
        QScopedValueRollback,
        QScreen,
        QScreenCursor,
        QScreenDriverFactory,
        QScreenDriverPlugin,
        QScriptable,
        QScriptClass,
        QScriptClassPropertyIterator,
        QScriptContext,
        QScriptContextInfo,
        QScriptEngine,
        QScriptEngineAgent,
        QScriptEngineDebugger,
        QScriptExtensionPlugin,
        QScriptProgram,
        QScriptString,
        QScriptSyntaxCheckResult,
        QScriptValue,
        QScriptValueIterator,
        QScrollArea,
        QScrollBar,
        SeekSlider (Phonon),
        QSemaphore,
        QSequentialAnimationGroup,
        QSessionManager,
        QSet,
        QSetIterator,
        QSettings,
        QSharedData,
        QSharedDataPointer,
        QSharedMemory,
        QSharedPointer,
        QShortcut,
        QShortcutEvent,
        QShowEvent,
        QSignalMapper,
        QSignalSpy,
        QSignalTransition,
        QSimpleXmlNodeModel,
        QSize,
        QSizeF,
        QSizeGrip,
        QSizePolicy,
        QSlider,
        QSocketNotifier,
        QSortFilterProxyModel,
        QSound,
        QSourceLocation,
        QSpacerItem,
        QSpinBox,
        QSplashScreen,
        QSplitter,
        QSplitterHandle,
        QSqlDatabase,
        QSqlDriver,
        QSqlDriverCreator,
        QSqlDriverCreatorBase,
        QSqlDriverPlugin,
        QSqlError,
        QSqlField,
        QSqlIndex,
        QSqlQuery,
        QSqlQueryModel,
        QSqlRecord,
        QSqlRelation,
        QSqlRelationalDelegate,
        QSqlRelationalTableModel,
        QSqlResult,
        QSqlTableModel,
        QSslCertificate,
        QSslCipher,
        QSslConfiguration,
        QSslError,
        QSslKey,
        QSslSocket,
        QStack,
        QStackedLayout,
        QStackedWidget,
        QStandardItem,
        QStandardItemEditorCreator,
        QStandardItemModel,
        QState,
        QStateMachine,
        QStaticText,
        QStatusBar,
        QStatusTipEvent,
        QString,
        QStringList,
        QStringListModel,
        QStringMatcher,
        QStringRef,
        QStyle,
        QStyledItemDelegate,
        QStyleFactory,
        QStyleHintReturn,
        QStyleHintReturnMask,
        QStyleHintReturnVariant,
        QStyleOption,
        QStyleOptionButton,
        QStyleOptionComboBox,
        QStyleOptionComplex,
        QStyleOptionDockWidget,
        QStyleOptionFocusRect,
        QStyleOptionFrame,
        QStyleOptionFrameV2,
        QStyleOptionFrameV3,
        QStyleOptionGraphicsItem,
        QStyleOptionGroupBox,
        QStyleOptionHeader,
        QStyleOptionMenuItem,
        QStyleOptionProgressBar,
        QStyleOptionProgressBarV2,
        QStyleOptionQ3DockWindow,
        QStyleOptionQ3ListView,
        QStyleOptionQ3ListViewItem,
        QStyleOptionRubberBand,
        QStyleOptionSizeGrip,
        QStyleOptionSlider,
        QStyleOptionSpinBox,
        QStyleOptionTab,
        QStyleOptionTabBarBase,
        QStyleOptionTabBarBaseV2,
        QStyleOptionTabV2,
        QStyleOptionTabV3,
        QStyleOptionTabWidgetFrame,
        QStyleOptionTabWidgetFrameV2,
        QStyleOptionTitleBar,
        QStyleOptionToolBar,
        QStyleOptionToolBox,
        QStyleOptionToolBoxV2,
        QStyleOptionToolButton,
        QStyleOptionViewItem,
        QStyleOptionViewItemV2,
        QStyleOptionViewItemV3,
        QStyleOptionViewItemV4,
        QStylePainter,
        QStylePlugin,
        QSupportedWritingSystems,
        QSvgGenerator,
        QSvgRenderer,
        QSvgWidget,
        QSwipeGesture,
        QSymbianEvent,
        QSymbianGraphicsSystemHelper,
        QSyntaxHighlighter,
        QSysInfo,
        QSystemLocale,
        QSystemSemaphore,
        QSystemTrayIcon,
        QTabBar,
        QTabletEvent,
        QTableView,
        QTableWidget,
        QTableWidgetItem,
        QTableWidgetSelectionRange,
        QTabWidget,
        QTapAndHoldGesture,
        QTapGesture,
        QTcpServer,
        QTcpSocket,
        QTemporaryFile,
        QTestEventList,
        QTextBlock,
        QTextBlockFormat,
        QTextBlockGroup,
        QTextBlockUserData,
        QTextBoundaryFinder,
        QTextBrowser,
        QTextCharFormat,
        QTextCodec,
        QTextCodecPlugin,
        QTextCursor,
        QTextDecoder,
        QTextDocument,
        QTextDocumentFragment,
        QTextDocumentWriter,
        QTextEdit,
        QTextEncoder,
        QTextFormat,
        QTextFragment,
        QTextFrame,
        QTextFrameFormat,
        QTextImageFormat,
        QTextInlineObject,
        QTextItem,
        QTextLayout,
        QTextLength,
        QTextLine,
        QTextList,
        QTextListFormat,
        QTextObject,
        QTextObjectInterface,
        QTextOption,
        QTextStream,
        QTextTable,
        QTextTableCell,
        QTextTableCellFormat,
        QTextTableFormat,
        QThread,
        QThreadPool,
        QThreadStorage,
        QTileRules,
        QTime,
        QTimeEdit,
        QTimeLine,
        QTimer,
        QTimerEvent,
        QToolBar,
        QToolBox,
        QToolButton,
        QToolTip,
        QTouchEvent,
        QTransform,
        QTranslator,
        QTreeView,
        QTreeWidget,
        QTreeWidgetItem,
        QTreeWidgetItemIterator,
        QUdpSocket,
        QUiLoader,
        QUndoCommand,
        QUndoGroup,
        QUndoStack,
        QUndoView,
        UnhandledException (QtConcurrent),
        QUrl,
        QUrlInfo,
        QUuid,
        QValidator,
        QVariant,
        QVariantAnimation,
        QVarLengthArray,
        QVBoxLayout,
        QVector,
        QVector2D,
        QVector3D,
        QVector4D,
        QVectorIterator,
        QVideoFrame,
        VideoPlayer (Phonon),
        QVideoSurfaceFormat,
        VideoWidget (Phonon),
        VideoWidgetInterface44 (Phonon),
        VolumeSlider (Phonon),
        QWaitCondition,
        QWeakPointer,
        QWebDatabase,
        QWebElement,
        QWebElementCollection,
        QWebFrame,
        QWebHistory,
        QWebHistoryInterface,
        QWebHistoryItem,
        QWebHitTestResult,
        QWebInspector,
        QWebPage,
        QWebPluginFactory,
        QWebSecurityOrigin,
        QWebSettings,
        QWebView,
        QWhatsThis,
        QWhatsThisClickedEvent,
        QWheelEvent,
        QWidget,
        QWidgetAction,
        QWidgetItem,
        QWindowsMime,
        QWindowsStyle,
        QWindowStateChangeEvent,
        QWindowsVistaStyle,
        QWindowsXPStyle,
        QWizard,
        QWizardPage,
        QWriteLocker,
        QWSCalibratedMouseHandler,
        QWSClient,
        QWSEmbedWidget,
        QWSEvent,
        QWSGLWindowSurface,
        QWSInputMethod,
        QWSKeyboardHandler,
        QWSMouseHandler,
        QWSPointerCalibrationData,
        QWSScreenSaver,
        QWSServer,
        QWSWindow,
        QX11EmbedContainer,
        QX11EmbedWidget,
        QX11Info,
        QXmlAttributes,
        QXmlContentHandler,
        QXmlDeclHandler,
        QXmlDefaultHandler,
        QXmlDTDHandler,
        QXmlEntityResolver,
        QXmlErrorHandler,
        QXmlFormatter,
        QXmlInputSource,
        QXmlItem,
        QXmlLexicalHandler,
        QXmlLocator,
        QXmlName,
        QXmlNamePool,
        QXmlNamespaceSupport,
        QXmlNodeModelIndex,
        QXmlParseException,
        QXmlQuery,
        QXmlReader,
        QXmlResultItems,
        QXmlSchema,
        QXmlSchemaValidator,
        QXmlSerializer,
        QXmlSimpleReader,
        QXmlStreamAttribute,
        QXmlStreamAttributes,
        QXmlStreamEntityDeclaration,
        QXmlStreamEntityResolver,
        QXmlStreamNamespaceDeclaration,
        QXmlStreamNotationDeclaration,
        QXmlStreamReader,
        QXmlStreamWriter
    },
}
\lstdefinelanguage{qml}
{
    keywords={typeof, new, true, false, catch, function, return, null, catch, switch, var, if, in, while, do, else, 
    case, break, import},
    keywords=[2]{class, export, boolean, throw, implements, import, this},
    sensitive=true,
    comment=[l]{//},
    morecomment=[s]{/*}{*/},
    morestring=[b]',
    morestring=[b]",
}


% \lstset{
%     literate={0}{{\textcolor{listingsnumbercolor}{0}}}{1}%
%              {1}{{\textcolor{listingsnumbercolor}{1}}}{1}%
%              {2}{{\textcolor{listingsnumbercolor}{2}}}{1}%
%              {3}{{\textcolor{listingsnumbercolor}{3}}}{1}%
%              {4}{{\textcolor{listingsnumbercolor}{4}}}{1}%
%              {5}{{\textcolor{listingsnumbercolor}{5}}}{1}%
%              {6}{{\textcolor{listingsnumbercolor}{6}}}{1}%
%              {7}{{\textcolor{listingsnumbercolor}{7}}}{1}%
%              {8}{{\textcolor{listingsnumbercolor}{8}}}{1}%
%              {9}{{\textcolor{listingsnumbercolor}{9}}}{1}%
%              {.0}{{\textcolor{listingsnumbercolor}{.0}}}{2}% Following is to ensure that only periods
%              {.1}{{\textcolor{listingsnumbercolor}{.1}}}{2}% followed by a digit are changed.
%              {.2}{{\textcolor{listingsnumbercolor}{.2}}}{2}%
%              {.3}{{\textcolor{listingsnumbercolor}{.3}}}{2}%
%              {.4}{{\textcolor{listingsnumbercolor}{.4}}}{2}%
%              {.5}{{\textcolor{listingsnumbercolor}{.5}}}{2}%
%              {.6}{{\textcolor{listingsnumbercolor}{.6}}}{2}%
%              {.7}{{\textcolor{listingsnumbercolor}{.7}}}{2}%
%              {.8}{{\textcolor{listingsnumbercolor}{.8}}}{2}%
%              {.9}{{\textcolor{listingsnumbercolor}{.9}}}{2}%
% }

\RequirePackage{calc}

%Define a reference depth. 
%You can choose either relative or absolute.
%--------------------------
\newlength{\DepthReference}
% \settodepth{\DepthReference}{g}%relative to a depth of a letter.
\setlength{\DepthReference}{0pt}%absolute value.

%Define a reference Height. 
%You can choose either relative or absolute.
%--------------------------
\newlength{\HeightReference}
% \settoheight{\HeightReference}{T}
\setlength{\HeightReference}{7pt}

\newlength{\Width}%
\newcommand{\lstinlinebox}[1]{%
  \settowidth{\Width}{\lstinline|#1|}%
  \fcolorbox{listingsrulecolor}{listingsbackgroundcolor}{%
    \raisebox{-\DepthReference}%
    {%
          \parbox[b][\HeightReference+\DepthReference][c]{\Width}{\centering \lstinline|#1|}%
    }%
  }%
}%