\documentclass[norsk,a4paper,12pt]{article}
% if you want a single-column, remove reprint

% allows special characters (including æøå)
\usepackage[utf8]{inputenc}
\usepackage [norsk]{babel} %if you write norwegian
%\usepackage[english]{babel}  %if you write english


\usepackage{physics,amssymb}  % mathematical symbols (physics imports amsmath)
\usepackage{graphicx}         % include graphics such as plots
\usepackage{xcolor}           % set colors
\usepackage{hyperref}         % automagic cross-referencing (this is GODLIKE)
\usepackage{tikz}             % draw figures manually
\usepackage{listings}         % display code
\usepackage{subfigure}        % imports a lot of cool and useful figure commands
\usepackage{float}			  % force placement of tables and figures
\usepackage{amsmath}
\usepackage{minted} %code

\hypersetup{ % this is just my personal choice, feel free to change things
	colorlinks,
	linkcolor={red!50!black},
	citecolor={blue!50!black},
	urlcolor={blue!80!black}}

%% Defines the style of the programming listing
%% This is actually my personal template, go ahead and change stuff if you want
\lstset{ %
	inputpath=,
	backgroundcolor=\color{white!88!black},
	basicstyle={\ttfamily\scriptsize},
	commentstyle=\color{magenta},
	language=Python,
	morekeywords={True,False},
	tabsize=4,
	stringstyle=\color{green!55!black},
	frame=single,
	keywordstyle=\color{blue},
	showstringspaces=false,
	columns=fullflexible,
	keepspaces=true}

\title{FYS3110 - Oblig 3 - Karl Henrik Fredly}

\begin{document}
	
	\maketitle
	
\section*{Problem 7.5(H)}

	Antallet symmetriske og anti-symmetriske spin tilstander for to partikler med spin-s er gitt ved trekanttallene:
	\begin{equation}
	\begin{aligned}
	T_n = \frac{1}{2} n (n + 1)
	\end{aligned}
	\end{equation}
	For to partikler med spin-s er antall symmetriske tilstander gitt ved $T_{2s + 1}$, og antall anti-symmetriske tilstander gitt ved $T_{2s}$.

	
\section*{Problem 7.6(H)}

	Ser over tabellen for Clebsch-Gordan koeffisientene og ser at to spin-1 bosoner har 6 spin tilstander som er symmetriske (5 for totalspinn 2, 1 for totalspinn 0), og 3 spin tilstander som er anti-symmetriske (for totalspinn 1).

	Bølgefunksjonene til harmonisk oscillator er gitt ved funksjonene $\psi_n(x)$ som har tilhørende energier $E_n = (n + \frac{1}{2})\hbar \omega$ for $n = 0,1,2,\dots$.
	
	Det laveste energinivået får vi når begge bosonene er i tilstand $\psi_0$, da får vi en symmetrisk romlig tilstand
	\begin{equation}
	\begin{aligned}
	\psi_0(x_1)\psi_0(x_2)
	\end{aligned}
	\end{equation}
	med energi $\hbar \omega$. Siden hele tilstanden skal bli symmetrisk (siden vi har to bosoner) har vi 6 mulige symmetriske spin tilstander, og dermed en degenerasjon på 6. 
	
	Det neste energinivået får vi når det ene bosonet er i tilstand $\psi_0$ og det andre i $\psi_1$. Da får vi en symmetrisk romlig tilstand
	\begin{equation}
	\begin{aligned}
	\frac{1}{\sqrt{2}}(\psi_0(x_1)\psi_1(x_2) + \psi_1(x_1)\psi_0(x_2))
	\end{aligned}
	\end{equation}
	med energi $2\hbar \omega$. Vi får også en antisymmetrisk romlig tilstand
	\begin{equation}
	\begin{aligned}
	\frac{1}{\sqrt{2}}(\psi_0(x_1)\psi_1(x_2) - \psi_1(x_1)\psi_0(x_2))
	\end{aligned}
	\end{equation}
	med energi $2\hbar \omega$. Den symmetriske romlige tilstanden har 6 mulige spinn tilstander, og den anti-symmetriske romlige tilstanden har 3 mulige spinn tilstander, så vi får totalt en degenerasjon på 9.
	
	Det neste energinivået får vi når begge bosonene er i tilstand $\psi_1$, da får vi en symmetrisk romlig tilstand
	\begin{equation}
	\begin{aligned}
	\psi_1(x_1)\psi_1(x_2)
	\end{aligned}
	\end{equation}
	med energi $3\hbar \omega$. Vi har 6 mulige symmetriske spin tilstander, og dermed en degenerasjon på 6.
	
\section*{Problem 7.7(H)}

	Ved tiden t er partikkelen i tilstanden $\ket{\psi(t)} = e^{-\frac{i}{\hbar}\hat{H}t} \ket{\psi(0)} = \hat{U} \ket{\psi(0)}$. Merk at $\hat{U}^\dagger \hat{U} = \hat{I}$. Hvis vi har en operator $\hat{Q}$ som kommuterer med $\hat{U}$ har vi
	\begin{equation}
	\begin{aligned}
	\bra{\psi(t)} \hat{Q} \ket{\psi(t)} &= \bra{\psi(0)} \hat{U}^\dagger \hat{Q} \hat{U} \ket{\psi(0)} \\
	&= \bra{\psi(0)} \hat{U}^\dagger \hat{U} \hat{Q} \ket{\psi(0)} = \bra{\psi(0)} \hat{Q} \ket{\psi(0)}
	\end{aligned}
	\end{equation}
	Altså er forventningsverdien til en størrelse gitt ved $\hat{Q}$ tidsuavhengig hvis $\hat{Q}$ kommuterer med $\hat{U}$, eller, siden $\hat{U}$ kan skrives som en kombinasjon av $\hat{H}$ potenser, hvis $\hat{Q}$ kommuterer med $\hat{H}$.
	
	Vi kan skrive
	\begin{equation}
	\begin{aligned}
	\hat{H} &= \frac{J}{\hbar} \vec{S_1} \cdot \vec{S_2} - \frac{h_1}{\hbar} S_1^z \\
	&= \frac{J}{\hbar} (S_1^x S_2^x + S_1^y S_2^y + S_1^z S_2^z) - \frac{h_1}{\hbar} S_1^z
	\end{aligned}
	\end{equation}
	
	For å finne ut hva som kommuterer med $\hat{H}$ regner vi ut
	\begin{equation}
	\begin{aligned}
	\comm{S_1^z}{\hat{H}} &= \comm{S_1^z}{\frac{J}{\hbar} (S_1^x S_2^x + S_1^y S_2^y + S_1^z S_2^z) - \frac{h_1}{\hbar} S_1^z} \\
	&= \comm{S_1^z}{\frac{J}{\hbar} (S_1^x S_2^x + S_1^y S_2^y)} \\
	&= i\hbar \frac{J}{\hbar} (S_1^y S_2^x - S_1^x S_2^y)
	\end{aligned}
	\end{equation}
	\begin{equation}
	\begin{aligned}
	\comm{S_2^z}{\hat{H}} &= \comm{S_1^z}{\frac{J}{\hbar} (S_1^x S_2^x + S_1^y S_2^y + S_1^z S_2^z) - \frac{h_1}{\hbar} S_1^z} \\
	&= \comm{S_1^z}{\frac{J}{\hbar} (S_1^x S_2^x + S_1^y S_2^y)} \\
	&= i\hbar \frac{J}{\hbar} (S_1^x S_2^y - S_1^y S_2^x)
	\end{aligned}
	\end{equation}
	Vi ser at av alle uttrykkene $G_i(t)$ er det kun $G_3(t)$ som er tidsuavhengig siden vi klart ser at det kun er $(S_1^z + S_2^z)$ som kommuterer med $\hat{H}$.
	
\section*{Problem 7.8(X)}

	\subsection*{a)}
	Velger basisen $\ket{1} = \left[ \begin{matrix} 1 \\ 0 \\ 0 \end{matrix} \right]$, $\ket{2} = \left[ \begin{matrix} 0 \\ 1 \\ 0 \end{matrix} \right]$, $\ket{3} = \left[ \begin{matrix} 0 \\ 0 \\ 1 \end{matrix} \right]$
	
	Ser da at produktet $\ket{i}\bra{j}$ er matrisen med et 1-tall i posisjon ij, og nuller ellers. Setter da bare inn 1-tall på de gitte indeksene i uttrykket for H.
	\begin{equation}
	\begin{aligned}
	H = -g \left[ \begin{matrix} 0 & 1 & 1 \\ 1 & 0 & 1 \\ 1 & 1 & 0 \end{matrix} \right]
	\end{aligned}
	\end{equation}
	
	\subsection*{b)}
	På samme måte som i a) finner vi
	\begin{equation}
	\begin{aligned}
	R = \left[ \begin{matrix} 0 & 0 & 1 \\ 1 & 0 & 0 \\ 0 & 1 & 0 \end{matrix} \right]
	\end{aligned}
	\end{equation}
	ser at denne matrisen har egenskapen $R\ket{1} = \ket{2}$, $R\ket{2} = \ket{3}$, $R\ket{3} = \ket{1}$, den er derfor en slags translasjons/rotasjonssymmetri der partikkelen blir sendt til ''neste'' atom. Den er ikke hermitisk, siden den transponerte og komplekskonjugerte klart gir en helt annen matrise. Den er unitær siden $R R^\dagger = I$.
	
	\subsection*{c)}
	Vi såg i b) at R sender egenkettene til neste egenket i rekka. $R^3$ sender derfor egenkettene til seg selv, og da må den være identitetsmatrisen, siden matriserepresentasjonen til en lineæravbilding er gitt av hvor den avbilder standardbasisvektorene. Vi får da
	
	\begin{equation}
	\begin{aligned}
	R^3 \ket{\psi} = \lambda^3 \ket{\psi} = \ket{\psi}
	\end{aligned}
	\end{equation}
	Vi får da egenverdiene $\lambda_1 = 1$, $\lambda_2 = \frac{-1 + \sqrt{3}i}{2}$ og $\lambda_3 = \frac{-1 - \sqrt{3}i}{2}$.
	
	Finner tilhørende egenvektorer
	\begin{equation}
	\begin{aligned}
	v_1 = \left[ \begin{matrix} 1 \\ 1 \\ 1 \end{matrix} \right], 
	v_2 = \left[ \begin{matrix} \frac{-1 + \sqrt{3}i}{2} \\ \frac{-1 - \sqrt{3}i}{2} \\ 1 \end{matrix} \right], 
	v_3 = \left[ \begin{matrix} \frac{-1 - \sqrt{3}i}{2} \\ \frac{-1 + \sqrt{3}i}{2} \\ 1 \end{matrix} \right]
	\end{aligned}
	\end{equation}
	
	H har egenverdiene $\lambda_1 = g$ og $\lambda_2 = -2g$.
	
	Finner tilhørende egenvektorer
	\begin{equation}
	\begin{aligned}
	v_{1,1} = \left[ \begin{matrix} -1 \\ 1 \\ 0 \end{matrix} \right], 
	v_{1,2} = \left[ \begin{matrix} -1 \\ 0 \\ 1 \end{matrix} \right], 
	v_2 = \left[ \begin{matrix} 1 \\ 1 \\ 1 \end{matrix} \right]
	\end{aligned}
	\end{equation}
	
	\subsection*{d)}
	Vi skriver $\ket{2}$ som en lineærkombinasjon av egentilstandene til $\hat{H}$:
	\begin{equation}
	\begin{aligned}
	\ket{2} = \left[ \begin{matrix} 0 \\ 1 \\ 0 \end{matrix} \right] = \frac{2}{3}(v_{1,1} - 0.5v_{1,2} + 0.5 v_2)
	\end{aligned}
	\end{equation}
	
	Partikkelen blir beskrevet av tilstanden
	\begin{equation}
	\begin{aligned}
	e^{-\frac{i}{\hbar}\hat{H}t} \ket{2} &= e^{-\frac{i}{\hbar}\hat{H}t} \frac{2}{3}(v_{1,1} - 0.5v_{1,2} + 0.5 v_2) \\
	&= \frac{2}{3}(e^{-\frac{i}{\hbar}\hat{H}t} v_{1,1} - 0.5e^{-\frac{i}{\hbar}\hat{H}t} v_{1,2} + 0.5e^{-\frac{i}{\hbar}\hat{H}t} v_{2}) \\
	&= \frac{2}{3}(e^{-\frac{i}{\hbar}gt} v_{1,1} - 0.5e^{-\frac{i}{\hbar}gt} v_{1,2} + 0.5e^{2\frac{i}{\hbar}gt} v_{2}) \\
	&= \frac{2}{3}(e^{-\frac{i}{\hbar}gt} (v_{1,1} - 0.5v_{1,2}) + 0.5e^{2\frac{i}{\hbar}gt} v_{2}) \\
	&= \frac{2}{3}(e^{-\frac{i}{\hbar}gt} (\left[ \begin{matrix} -1 \\ 1 \\ 0 \end{matrix} \right] - 0.5\left[ \begin{matrix} -1 \\ 0 \\ 1 \end{matrix} \right]) + 0.5e^{2\frac{i}{\hbar}gt} \left[ \begin{matrix} 1 \\ 1 \\ 1 \end{matrix} \right])
	\end{aligned}
	\end{equation}
	
	Sannsynligheten for å finne partikkelen ved atom 2 etter en tid t er:
	\begin{equation}
	\begin{aligned}
	|\bra{2} e^{-\frac{i}{\hbar}\hat{H}t} \ket{2}|^2 &= |\frac{2}{3}(e^{-\frac{i}{\hbar}gt} (1 - 0) + 0.5e^{2\frac{i}{\hbar}gt})|^2 = |\frac{2}{3}(e^{-\frac{i}{\hbar}gt} + 0.5e^{2\frac{i}{\hbar}gt})|^2
	\end{aligned}
	\end{equation}

	
	
\end{document}