\documentclass[norsk,a4paper,12pt]{article}
% if you want a single-column, remove reprint

% allows special characters (including æøå)
\usepackage[utf8]{inputenc}
\usepackage [norsk]{babel} %if you write norwegian
%\usepackage[english]{babel}  %if you write english


\usepackage{physics,amssymb}  % mathematical symbols (physics imports amsmath)
\usepackage{graphicx}         % include graphics such as plots
\usepackage{xcolor}           % set colors
\usepackage{hyperref}         % automagic cross-referencing (this is GODLIKE)
\usepackage{tikz}             % draw figures manually
\usepackage{listings}         % display code
\usepackage{subfigure}        % imports a lot of cool and useful figure commands
\usepackage{float}			  % force placement of tables and figures
\usepackage{amsmath}
\usepackage{minted} %code

\hypersetup{ % this is just my personal choice, feel free to change things
	colorlinks,
	linkcolor={red!50!black},
	citecolor={blue!50!black},
	urlcolor={blue!80!black}}

%% Defines the style of the programming listing
%% This is actually my personal template, go ahead and change stuff if you want
\lstset{ %
	inputpath=,
	backgroundcolor=\color{white!88!black},
	basicstyle={\ttfamily\scriptsize},
	commentstyle=\color{magenta},
	language=Python,
	morekeywords={True,False},
	tabsize=4,
	stringstyle=\color{green!55!black},
	frame=single,
	keywordstyle=\color{blue},
	showstringspaces=false,
	columns=fullflexible,
	keepspaces=true}

\title{FYS3110 - Oblig 2 - Karl Henrik Fredly}

\begin{document}
	
	\maketitle
	
	\section*{Problem 2.6(X)}
	
		En hermitisk operator er en operator som er lik sin egen hermitisk konjugerte. Dvs.
		\begin{equation}
		\begin{aligned}
		\hat{O} = \hat{O}^\dagger \iff \bra{\psi}\hat{O}\ket{\phi} = \bra{\phi}\hat{O}\ket{\psi}^*
		\end{aligned}
		\end{equation}
		Hermitiske operatorer passer godt til å representere fysiske observable fordi de kun har reelle egenverdier.
	
	\section*{Problem 2.7(X)}
	
		\begin{equation}
		\begin{aligned}
		\bra{\psi}\hat{O}\ket{\phi} = \bra{\psi}\hat{A}\hat{B}\ket{\phi}
		= \bra{\hat{A}^\dagger\psi}\hat{B}\ket{\phi} = \bra{\hat{B}^\dagger\hat{A}^\dagger\psi}\ket{\phi}
		= \bra{\hat{O}^\dagger\psi}\ket{\phi}
		\Rightarrow \hat{O} = \hat{B}^\dagger\hat{A}^\dagger
		\end{aligned}
		\end{equation}
		$\hat{B}\ket{\phi}$ er en vektor, så man kan flytte $\hat{A}$ alene inn i ket-en først.
	
		Hvis $\hat{O}$, $\hat{A}$, og $\hat{B}$ er hermitiske har vi
		
		\begin{equation}
		\begin{aligned}
		\comm{\hat{A}}{\hat{B}} = \hat{A}\hat{B} - \hat{B}\hat{A} = \hat{A}\hat{B} - \hat{B}^\dagger\hat{A}^\dagger
		= \hat{O} - \hat{O}^\dagger = \hat{O} - \hat{O} = 0
		\end{aligned}
		\end{equation}
		
	\section*{Problem 2.8(X)}
		
		Vi uttrykker $\hat{K}$ ved operatorene $\hat{x}$ og $\hat{p}$ siden vi vet at disse er hermitiske
		\begin{equation}
		\begin{aligned}
		\hat{K} = x\frac{d}{dx} = \frac{1}{-i\hbar}x (-i\hbar\frac{d}{dx}) = \frac{1}{-i\hbar}\hat{x}\hat{p}
		\end{aligned}
		\end{equation}
		Da finner vi $\hat{K}^\dagger$ ved regler for hermitisk konjugerte
		\begin{equation}
		\begin{aligned}
		\hat{K}^\dagger = (\frac{1}{-i\hbar}\hat{x}\hat{p})^\dagger = (\frac{1}{-i\hbar})^*\hat{p}^\dagger\hat{x}^\dagger
		= (\frac{1}{i\hbar})\hat{p}\hat{x} = \frac{1}{i\hbar}(-i\hbar\frac{d}{dx})x 
		= -\frac{d}{dx}x
		\end{aligned}
		\end{equation}
		
		Vi kan også gjøre det på en mer tungvindt måte:
		
		\begin{equation}
		\begin{aligned}
		\bra{\psi}\hat{K}^\dagger\ket{\phi} &= \bra{\phi}\hat{K}\ket{\psi}^* \\
		&= (\int_{-\infty}^{\infty}\phi^*(x)\hat{K}\psi(x)dx)^* \\
		&= \int_{-\infty}^{\infty}\phi(x)x\frac{d}{dx}\psi^*(x)dx \\
		&= \int_{-\infty}^{\infty}u\frac{dv}{dx}dx
		\end{aligned}
		\end{equation}
		der $u = \phi(x)x$ og $\frac{dv}{dx} = \frac{d}{dx}\psi^*(x)$
		\begin{equation}
		\begin{aligned}
		\bra{\psi}\hat{K}^\dagger\ket{\phi} &= \int_{-\infty}^{\infty}u\frac{dv}{dx}dx \\
		&= [uv]_{-\infty}^{\infty} - \int_{-\infty}^{\infty}v\frac{du}{dx}dx \\
		&= [x\phi(x)\psi^*(x)]_{-\infty}^{\infty} - \int_{-\infty}^{\infty}\psi^*(x)\frac{d}{dx}x\phi(x)dx \\
		&= 0 + \int_{-\infty}^{\infty}\psi^*(x)(-\frac{d}{dx}x)\phi(x)dx \\
		&= \bra{\psi}\hat{K}^\dagger\ket{\phi} \Rightarrow \hat{K}^\dagger = -\frac{d}{dx}x
		\end{aligned}
		\end{equation}
	\section*{Problem 2.9(H)}
		
	\subsection*{a)}
		
		\begin{equation}
		\begin{aligned}
		\bra{\psi} = c^*(\sqrt{7} \bra{0} + i \bra{1})^* = c^*(\sqrt{7} \bra{0} - i \bra{1})
		\end{aligned}
		\end{equation}
		
		\begin{equation}
		\begin{aligned}
		\bra{\psi}\ket{\psi} &= cc^*(\sqrt{7}\sqrt{7} \bra{0}\ket{0} + i(-i) \bra{0}\ket{0})
		= cc^*(\sqrt{7}\sqrt{7} + 1) = 8cc^* = 1 \\
		&\Rightarrow cc^* = \frac{1}{8} \Rightarrow \abs{c} = \frac{1}{\sqrt{8}}
		\end{aligned}
		\end{equation}
	
	\subsection*{b)}
	
		\begin{equation}
		\begin{aligned}
		\ket{\psi} = c(\sqrt{7} \ket{0} + i \ket{1})
		= c(\sqrt{7} \begin{pmatrix}1 \\ 0 \end{pmatrix}  + i \begin{pmatrix}0 \\ 1 \end{pmatrix})
		= c(\begin{pmatrix}\sqrt{7} \\ 0 \end{pmatrix}  + \begin{pmatrix}0 \\ i \end{pmatrix})
		= c\begin{pmatrix}\sqrt{7} \\ i \end{pmatrix}
		\end{aligned}
		\end{equation}
		
		Kolonnene til $\hat{A}$ svarer til hvor den avbilder standardbasisvektorene $\ket{0}$ og $\ket{1}$.
		
		\begin{equation}
		\begin{aligned}
		\hat{A} = \begin{pmatrix}0 & i \\ -i & 0 \end{pmatrix}
		\end{aligned}
		\end{equation}
		
	\subsection*{c)}
	
		\begin{equation}
		\begin{aligned}
		\bra{\psi} \hat{A} \ket{\psi} &= cc^* \begin{pmatrix}\sqrt{7} & -i \end{pmatrix}
		\begin{pmatrix}0 & i \\ -i & 0 \end{pmatrix} \begin{pmatrix}\sqrt{7} \\ i \end{pmatrix} \\
		&= cc^* \begin{pmatrix}\sqrt{7} & -i \end{pmatrix}
		\begin{pmatrix}i^2 \\ -i\sqrt{7} \end{pmatrix} \\
		&= cc^* (-\sqrt{7} -\sqrt{7}) = \frac{-2\sqrt{7}}{8} = \frac{-\sqrt{7}}{4}
		\end{aligned}
		\end{equation}
		
		\begin{equation}
		\begin{aligned}
		\bra{\psi} \hat{A} \ket{\psi} &= c(\bra{\psi} \hat{A} (\sqrt{7} \ket{0} + i \ket{1})) \\
		&= c(\bra{\psi} (-i\sqrt{7} \ket{1} + i^2 \ket{0})) \\
		&= cc^*((\sqrt{7} \bra{0} - i \bra{1}) (-i\sqrt{7} \ket{1} + i^2 \ket{0})) \\
		&= cc^*(i^2\sqrt{7} + i^2\sqrt{7}) = cc^*(-\sqrt{7} - \sqrt{7}) = \frac{-\sqrt{7}}{4}
		\end{aligned}
		\end{equation}
		
	\section*{Problem 2.10(H)}
	
	\subsection*{a)}
	
		\begin{equation}
		\begin{aligned}
		U = \begin{pmatrix} a & b \\ c & d \end{pmatrix}, U^T = \begin{pmatrix} a & c \\ b & d \end{pmatrix}
		, U^\dagger = \begin{pmatrix} a^* & c^* \\ b^* & d^* \end{pmatrix}
		\end{aligned}
		\end{equation}
		
	\subsection*{b)}
	
		For at U skal hære hermitisk må $U = U^\dagger$, altså må $a = a^*$, $b = c^*$, $c = b^*$ og $d = d^*$.
		
	\subsection*{c)}
	
		Finner egenverdier ved å trekke fra $\lambda I$ og sette determinanten lik 0
		\begin{equation}
		\begin{aligned}
		\begin{vmatrix} a - \lambda & b \\ c & d - \lambda \end{vmatrix} &= 0 \\
		(a - \lambda)(d - \lambda) - bc &= 0 \\
		\lambda^2 - a\lambda - d\lambda - ad - bc &= 0 \\
		\lambda^2 + \lambda(-a - d) + ( ad - bc) &= 0 \\
		\lambda^2 + \lambda x + y &= 0
		\end{aligned}
		\end{equation}
		finner $\lambda$ med abc (med a = 1, b = x og y = c)
		\begin{equation}
		\begin{aligned}
		\lambda &= \frac{-x \pm \sqrt{x^2 - 4y}}{2} \\
		&= \frac{-(-a - d) \pm \sqrt{(-a - d)^2 - 4( ad - bc)}}{2} \\
		&= \frac{a + d \pm \sqrt{a^2 + d^2 + 2ad - 4ad + 4bc}}{2} \\
		&= \frac{a + d \pm \sqrt{a^2 + d^2 - 2ad + 4bc}}{2} \\
		&= \frac{a + d \pm \sqrt{(a - d)^2 + 4bc}}{2} \\
		\end{aligned}
		\end{equation}
		Når U er hermitisk er a og d reelle siden a = a$^*$ og d = d$^*$. Derfor er (a - d)$^2$ reellt og ikke-negativt. Når U er hermitisk er $bc = bb^* = \abs{b}^2$ som er reellt og ikke-negativt. Da blir uttrykket under rottegnet reellt og ikke-negativt. Derfor er egenverdiene til U reelle når U er hermitisk.
		
	\subsection*{d)}
	
		Når U er unitær og hermitisk er $UU^\dagger = UU = I$
		\begin{equation}
		\begin{aligned}
		UU &= I \\
		\begin{pmatrix} a & b \\ c & d \end{pmatrix} \begin{pmatrix} a & b \\ c & d \end{pmatrix} &= \begin{pmatrix} 1 & 0 \\ 0 & 1 \end{pmatrix}
		\end{aligned}
		\end{equation}
		Som gir disse ligningene
		\begin{equation}
		\begin{aligned}
		a^2 +bc = 1 \\
		ab + b^2 = 0 \\
		ac + dc = 0 \\
		bc + d^2 = 0
		\end{aligned}
		\end{equation}
		Vi ser at (ligning 1 og 4) $a^2 = d^2$. Og at(ligning 2) $b(a + b) = 0$ som gir $a = -b$ eller $b = 0$. Ligning 3 gir $c(a + d) = 0$ som gir $a = -d$ og/eller $c = 0$. Vi vet også at a og d er reelle, og at $b = c^*$, som gir (ligning 1) at $\abs{a} \leq 1$.
		
		Det er uendelig mange verdier for a, b, c og d som oppfyller disse kravene.
		
	\subsection*{e)}
		
		Når U er unitær og hermitisk er $UU^\dagger = UU = I$. La v være en egenvektor til U med tilhørende egenverdi $\lambda$.
		\begin{equation}
		\begin{aligned}
		UU^\dagger v = Iv = v = UUv = \lambda^2 v 
		\end{aligned}
		\end{equation}
		Vi ser at $\lambda^2 = 1$, dvs. $\lambda = \pm 1$.
		
	\section*{Problem 2.11(H)}
	
		En operator er hermitisk hvis $\bra{u}\hat{O}\ket{v} = \bra{v}\hat{O}\ket{u}^*$ for vilkårlige vektorer u og v. Vi kan skrive en vilkårlig vektor som $u = a\ket{\psi} + b\ket{\phi} + \sum_{n=1}^{N}c_n \ket{\gamma_n}, a,b,c_n \in \real$, siden $\ket{\psi}$, $\ket{\phi}$ og alle $\ket{\gamma_n}$ er lineært uavhengige og danner en komplett basis.
		
		Tilsvarende skriver vi en vilkårlig vektor $v = x\ket{\psi} + y\ket{\phi} + \sum_{n=1}^{N}z_n \ket{\gamma_n}, x,y,z_n \in \real$.
		
		Regner nå ut $\bra{u}\hat{H}\ket{v}$
		
		\begin{equation}
		\begin{aligned}
		\bra{u}\hat{H}\ket{v} &= \bra{u}\hat{H}(x\ket{\psi} + y\ket{\phi} + \sum_{n=1}^{N}z_n \ket{\gamma_n}) \\
		&= \bra{u}(xg\ket{\phi} + yg^*\ket{\psi}) \\
		&= (a\bra{\psi} + b\bra{\phi} + \sum_{n=1}^{N}c_n \bra{\gamma_n})(xg\ket{\phi} + yg^*\ket{\psi}) \\
		&= (a\bra{\psi} + b\bra{\phi})(xg\ket{\phi} + yg^*\ket{\psi}) \\
		&= ayg^*\bra{\psi}\ket{\psi} + bxg\bra{\phi}\ket{\phi} + axg\bra{\psi}\ket{\phi} + byg^*\bra{\phi}\ket{\psi} \\
		&= ayg^* + bxg + axg\bra{\psi}\ket{\phi} + byg^*\bra{\phi}\ket{\psi}
		\end{aligned}
		\end{equation}
		og nå $\bra{v}\hat{H}\ket{u}^*$
		\begin{equation}
		\begin{aligned}
		\bra{v}\hat{H}\ket{u}^* &= (\bra{v}\hat{H}(a\ket{\psi} + b\ket{\phi} + \sum_{n=1}^{N}c_n \ket{\gamma_n}))^* \\
		&= (\bra{v}(ag\ket{\phi} + bg^*\ket{\psi}))^* \\
		&= ((x\bra{\psi} + y\bra{\phi} + \sum_{n=1}^{N}z_n \bra{\gamma_n})(ag\ket{\phi} + bg^*\ket{\psi}))^* \\
		&= ((x\bra{\psi} + y\bra{\phi})(ag\ket{\phi} + bg^*\ket{\psi}))^* \\
		&= (xbg^*\bra{\psi}\ket{\psi} + yag\bra{\phi}\ket{\phi} + xag\bra{\psi}\ket{\phi} + ybg^*\bra{\phi}\ket{\psi})^* \\
		&= (xbg^* + yag + xag\bra{\psi}\ket{\phi} + ybg^*\bra{\phi}\ket{\psi})^* \\
		&= xbg + yag^* + xag^*\bra{\phi}\ket{\psi} + ybg\bra{\psi}\ket{\phi}
		\end{aligned}
		\end{equation}
		Setter nå $\bra{u}\hat{H}\ket{v} = \bra{v}\hat{H}\ket{u}^*$
		\begin{equation}
		\begin{aligned}
		ayg^* + bxg + axg\bra{\psi}\ket{\phi} + byg^*\bra{\phi}\ket{\psi} &= xbg + yag^* + xag^*\bra{\phi}\ket{\psi} + ybg\bra{\psi}\ket{\phi} \\
		axg\bra{\psi}\ket{\phi} + byg^*\bra{\phi}\ket{\psi} &= axg^*\bra{\phi}\ket{\psi} + byg\bra{\psi}\ket{\phi}
		\end{aligned}
		\end{equation}
		$\hat{H}$ er hermitisk hvis likheten holder, som er hvis $\bra{\psi}\ket{\phi} = 0$ eller hvis $\bra{\psi}\ket{\phi} = g^*$.
	
\end{document}